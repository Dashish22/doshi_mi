\documentclass[12pt]{article}
\usepackage[utf8]{inputenc}
\usepackage[russian]{babel}
\usepackage{amsfonts}
\usepackage{hyperref} % Включить ссылки в PDF
\usepackage{amsmath}
\numberwithin{equation}{section} % Изменить нумерацию формул (1) -> (1.1)
\begin{document}
	\tableofcontents % Вставить содержание
	\newpage
	\section{Коды Рида-Соломона}
	
	Важный частный случай q-ичных БЧХ-кодов длины   $q^m$ -- 1 представляют кода с $m=1$. Такие кода называются кодами Рида-Соломона, так как они впервые были описаны И.Ридом и Г.Соломоном в публикации 1960 г. После появления в том же 1960 г. работы Н. Цилера и Д.Горенстейна, содержащей обобщение конструкции БЧХ-кодов на недвоичный код, оказалось, что коды Рида--Соломона являются частным случаем БЧХ-кодов.
	
	Легко видеть, что размерность $k$ $q$-ичного кода Рида-Соломона длины $n=q-1$ c конструктивным расстоянием $d$ равна $n-d+1$, т.е. длина, размерность и конструктивное расстояние кодов Рида--Соломона связаны равенством
	 \begin{equation}
	d=n-k+1
	 \label{eq.simple}
	\end{equation}
	
	
	В то же время известно следующая теорема, неравенство которой называется границей Синглтона.
	
	{\bf Теорема 16.4}. {\itshape Для любого $(n,k)$-кода с минимальным расстоянием $d$,справедливо неравенство $d \leq {n-k+1}$.}	
	
	{\scshape Доказательство}. Пусть $\mathbf H$ --- проверочная матрица линейного $(n,k)$-кода с минимальным расстоянием $d$. Ранг этой матрицы равен $n-k$, и при этом любые ее $d-1$ столбцов линейно независимы. Следовательно,  ${n-k}\geq {d-1}$. Теорема доказана.
	
	Из сравнения границы Синглтона и равенства (1.1) следует, что конструктивное расстояние кодов Рида-Соломона совпадает с их минимальным расстоянием, а сами коды являются максимальными.
	
	Найдем порождающий многочлен кода Рида-Соломона длины 15, исправляющего две ошибки. В качестве поля из 16 элементов возьмем поле $\mathbb {Z}_{2}$[$\alpha$], где   $\alpha$ -- корень примитивного многочлена $x^4 \oplus x \oplus 1$   . Напомним, что степни примитивного элемента $\alpha$  этого поля выглядят следующим образом:
	
    $\alpha^0=(0001)$,\hspace{1cm} $\alpha^4=(0011)$,\hspace{1cm} $\alpha^8=(0101)$,\hspace{1cm} $\alpha^{12}=(1111)$,
    
    $\alpha^1=(0010)$,\hspace{1cm} $\alpha^5=(0110)$,\hspace{1cm} $\alpha^9=(1010)$,\hspace{1cm} $\alpha^{13}=(1101)$,
    
    $\alpha^2=(0100)$,\hspace{1cm} $\alpha^6=(1100)$,\hspace{1cm} $\alpha^{10}=(0111)$,\hspace{1cm} $\alpha^{14}=(1001)$,
    
    $\alpha^3=(1000)$,\hspace{1cm} $\alpha^7=(1011)$,\hspace{1cm} $\alpha^{11}=(1110)$.
   

	
	 В этом случае многочлен
	 
	 $$h(x)= (x \oplus \alpha)(x \oplus \alpha^2)
	 (x \oplus \alpha^3)
	 (x \oplus \alpha^4)$$=
	 
	 $$x^4 \oplus(\alpha \oplus \alpha^2 \oplus \alpha^3 \oplus \alpha^4)x^3
	 \oplus(\alpha^3 \oplus \alpha^4 \oplus \alpha^6 \oplus \alpha^7)x^2 \oplus
	     (\alpha^6 \oplus \alpha^7 \oplus \alpha^8 \oplus \alpha^9)x \oplus \alpha^{10}$$=
	     
	     $$x^4 \oplus \alpha^{13}x^3 \oplus \alpha^6x^2 \oplus \alpha^3x \oplus \alpha^{10}$$
	     
	 
	 
	 будет искомым порождающим многочленом.
\end{document}