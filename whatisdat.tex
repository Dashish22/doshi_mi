\documentclass[12pt]{article}
\usepackage{ucs}
\usepackage[utf8x]{inputenc}
\usepackage[russian]{babel}
\begin{document}
   	Методы решения интегралов.
   	
   	1. Непосредственное интегрирование
   	
   	Непосредственное интегрирование - метод интегрирования, при котором подынтегральная функция путем тождественных преобразований и применения свойств интеграла приводится к одному или нескольким табличным интегралам.
   	
   	2.Метод подведения под знак дифференциала
   	
   	Этот метод является эквивалентным методу подстановки. Если $f(x) = $$v(u(x)), то 
   	
  $ 	$\displaystyle\int f(x) dx$ = $\displaystyle\int v(u(x)) dx$\cdot \frac{d(u(x))}{d(u(x))}=$\displaystyle\int v(u(x))$\cdot\frac{d(u(x))}{\frac{d(u(x))}{dx}}=$\displaystyle\int v(u(x))$
   	\cdot\frac{d(u(x))}{u'(x)} 
   	
   	3. Метод замены переменной или метод подстановки
   	
   Этот метод заключается во введении новой переменной интегрирования (то есть делается подстановка). При этом заданный интеграл приводится к новому интегралу, который является табличным или с помощью преобразований его можно свести к табличному. 
   	
   	Пусть требуется вычислить интеграл $\textstyle\int f(x)dx$. Сделаем подстановку $x=$ $\varphi$(t). Тогда $dx=$ $\varphi'(t) и интеграл принимает вид:
   	
   	$\displaystyle\int f(x)dx=$\displaystyle\int f($\varphi(t)\cdot$\varphi'(t)dt
   	
   	4.Метод интегрирования по частям
   	
   	Этот метод основывается на следующей формуле:
   	
    $	$\displaystyle\int udv$ = $ uv - $\displaystyle\int vdu$
   
   
   	или
   	
   	
   $	$\displaystyle\int u(x)v'(x)dx$ = $ u(x)v(x)-$ $\displaystyle\int v(x)u'(x)dx$
   	
   	При этом предполагается, что нахождение интеграла $\textstyle\int vdu$ проще, чем исходного интеграла $\textstyle\int udv$ . В противном случае применение метода неоправданно. 
   	
	
\end{document}
